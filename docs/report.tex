% !TEX TS-program = pdflatex
% !TEX encoding = UTF-8 Unicode

% This is a simple template for a LaTeX document using the "article" class.
% See "book", "report", "letter" for other types of document.

\documentclass[11pt,twocolumn]{article} % use larger type; default would be 10pt

\usepackage[utf8]{inputenc} % set input encoding (not needed with XeLaTeX)
\usepackage{amsmath}
\usepackage{epsfig}
\usepackage{listings}
\usepackage{appendix}

%%% Examples of Article customizations
% These packages are optional, depending whether you want the features they 
% provide. See the LaTeX Companion or other references for full information.

%%% PAGE DIMENSIONS
\usepackage{geometry} % to change the page dimensions
\geometry{a4paper} % or letterpaper (US) or a5paper or....
\geometry{margin=0.75in} % for example, change the margins to 2 inches all round
% \geometry{landscape} % set up the page for landscape
%   read geometry.pdf for detailed page layout information

\usepackage{graphicx} % support the \includegraphics command and options

% Activate to begin paragraphs with an empty line rather than an indent
% \usepackage[parfill]{parskip}

%%% PACKAGES
\usepackage{booktabs} % for much better looking tables
\usepackage{array} % for better arrays (eg matrices) in maths
\usepackage{paralist} % very flexible & customisable lists 
\usepackage{verbatim} % adds environment for commenting out blocks of text 
\usepackage{subfig} % include more than one figure in a float
% These packages are all incorporated in the memoir class...

%%% HEADERS & FOOTERS
\usepackage{fancyhdr} % This should be set AFTER setting up the page geometry
\pagestyle{fancy} % options: empty , plain , fancy
\renewcommand{\headrulewidth}{0pt} % customise the layout...
\lhead{}\chead{}\rhead{}
\lfoot{}\cfoot{\thepage}\rfoot{}

%%% SECTION TITLE APPEARANCE
\usepackage{sectsty}
\allsectionsfont{\sffamily\mdseries\upshape} % (See the fntguide.pdf)
% (This matches ConTeXt defaults)

%%% ToC (table of contents) APPEARANCE
\usepackage[nottoc,notlof,notlot]{tocbibind} % Put the bibliography in the ToC
\usepackage[titles,subfigure]{tocloft} % Alter the style of the ToC
\renewcommand{\cftsecfont}{\rmfamily\mdseries\upshape}
\renewcommand{\cftsecpagefont}{\rmfamily\mdseries\upshape} % No bold!

%%% END Article customizations

%%% The "real" document content comes below...

\title{COMP3032 Intelligent Altgorithms Coursework}
\author{Edward Seabrook 
\\ Electronics and Computer Science 
\\ University of Southampton }
%\date{} % Activate to display a given date or no date (if empty),
				 % otherwise the current date is printed 

\begin{document}

\twocolumn[
	\begin{@twocolumnfalse}
		\maketitle
		\begin{abstract}

			This is an abstract, it's kinda like modern art...

		\end{abstract}
	\end{@twocolumnfalse}
]

\section{Introduction}
Dynamic programming is a method of reducing a search space when finding an
optimum solution to a complex problem. It relies on the fact that the problem
can be broken up into smaller, simpler problems. 

\section{Method}
For this project, I chose to use Matlab as it provides good matrix and vector
manipulation functionality. 

When approaching this task, I broke it up into several scripts and functions,
each with a single clearly defined purpose.  

\subsection{Search Space Extraction}
In my Matlab code, the function \em GenerateSearchSpace \em takes in a number of
divisions, an image and two contours. It returns a matrix of vertices and a
matrix of intensities. This differs slightly from the original specification,
but the matrix of vertices offers an easy way to convert back to image space
from search space. 

The function then loops over each point in the contours. For each point on
the first contour, the distance from the corresponding point on the second
contour is calculated. This distance is then used to interpolate the correct
number of points to sample between the two lines. 

For each of these interpolated points, a sample is taken. Initially I
implemented a function \em BasicSample \em which simply rounded the point to
the nearest integer an used the value of image at this pixel. I later extended
this by creating a function \em MultiSample \em which was loosely based on a
Gaussian sampling and sampled the pixels surrounding the rounded pixel with
weights proportional to the distance from the point.

I then implemented 

\subsection{Dynamic Programming Search}

\section{Results}

\subsection{Scaling}

\subsection{Varying Llambda}

\subsection{Initialisation}

\subsection{Different Images}

\section{Conclusion}

\begin{thebibliography}{9}

	\bibitem{img}
		Flickr Image of a Vapour Trail, \\
		http://www.flickr.com/photos/charliebrewer/484937808/.

\end{thebibliography}

\end{document}
