% !TEX TS-program = pdflatex
% !TEX encoding = UTF-8 Unicode

% This is a simple template for a LaTeX document using the "article" class.
% See "book", "report", "letter" for other types of document.

\documentclass[11pt,twocolumn]{article} % use larger type; default would be 10pt

\usepackage[utf8]{inputenc} % set input encoding (not needed with XeLaTeX)
\usepackage{amsmath}
\usepackage{epsfig}
\usepackage{listings}
\usepackage{appendix}

\usepackage{url}
\usepackage{hyperref}


%%% Examples of Article customizations
% These packages are optional, depending whether you want the features they 
% provide. See the LaTeX Companion or other references for full information.

%%% PAGE DIMENSIONS
\usepackage[top=2.4cm, bottom=2.4cm, left=2.5cm, right=2.5cm]{geometry}
\geometry{a4paper} % or letterpaper (US) or a5paper or....
% \geometry{landscape} % set up the page for landscape
%   read geometry.pdf for detailed page layout information

\usepackage{graphicx} % support the \includegraphics command and options

% Activate to begin paragraphs with an empty line rather than an indent
% \usepackage[parfill]{parskip}

%%% PACKAGES
\usepackage{booktabs} % for much better looking tables
\usepackage{array} % for better arrays (eg matrices) in maths
\usepackage{paralist} % very flexible & customisable lists 
\usepackage{verbatim} % adds environment for commenting out blocks of text 
\usepackage{subfig} % include more than one figure in a float
% These packages are all incorporated in the memoir class...

%%% HEADERS & FOOTERS
\usepackage{fancyhdr} % This should be set AFTER setting up the page geometry
\pagestyle{fancy} % options: empty , plain , fancy
\renewcommand{\headrulewidth}{0pt} % customise the layout...
\lhead{}\chead{}\rhead{}
\lfoot{}\cfoot{\thepage}\rfoot{}

%%% SECTION TITLE APPEARANCE
\usepackage{sectsty}
\allsectionsfont{\sffamily\mdseries\upshape} % (See the fntguide.pdf)
% (This matches ConTeXt defaults)

%%% ToC (table of contents) APPEARANCE
\usepackage[nottoc,notlof,notlot]{tocbibind} % Put the bibliography in the ToC
\usepackage[titles,subfigure]{tocloft} % Alter the style of the ToC
\renewcommand{\cftsecfont}{\rmfamily\mdseries\upshape}
\renewcommand{\cftsecpagefont}{\rmfamily\mdseries\upshape} % No bold!

%%% END Article customizations

%%% The "real" document content comes below...

\title{COMP3032 Intelligent Algorithms Coursework}
\author{\href{ejfs1g10@ecs.soton.ac.uk}{Edward JF Seabrook} 
\\ Electronics and Computer Science 
\\ University of Southampton }
%\date{} % Activate to display a given date or no date (if empty),
				 % otherwise the current date is printed 

\begin{document}

\twocolumn[
	\begin{@twocolumnfalse}
		\maketitle
		\begin{abstract}

Dynamic programming can be used to perform contour detection in a feasible
period of time. In this project, a dynamic programming algorithm was created for
extracting a tongue from an ultrasound image. The algorithm was tested to see how it
scales with the number of divisions, how it performs with varying lambda and how
it performed of different images. 

		\end{abstract}
	\end{@twocolumnfalse}
]

\section{Introduction}
Dynamic programming is a method of reducing a search space when finding an
optimum solution to a complex problem. It relies on the fact that the problem
can be broken up into smaller, simpler problems. Compared to the brute force
method of evaluating all possible contours, dynamic programming is very
efficient.

\section{Method}
For this project, I chose to use Matlab as it provides good matrix and vector
manipulation functionality. When approaching the task, I broke it up into
several scripts and functions, each with a single clearly defined purpose, to
ensure clarity and ease of reuse.  

\subsection{Search Space Extraction}
In my Matlab code, the function \em GenerateSearchSpace \em takes in a number of
divisions, an image and two contours. It returns a matrix of vertices and a
matrix of intensities. This differs slightly from the original specification,
but the matrix of vertices offers an easy way to convert back to image space
from search space. 

The function then loops over each point in the contours. For each point on
the first contour, the distance from the corresponding point on the second
contour is calculated. This distance is then used to interpolate the correct
number of points to sample between the two lines. Example 
points can be seen in Appendix \ref{SearchSpaceExt}.


For each of these interpolated points, a sample is taken. Initially I
implemented a function \em BasicSample \em which simply rounded the point to
the nearest integer an used the value of image at this pixel. I later extended
this by creating a function \em MultiSample \em which was loosely based on a
Gaussian sampling and sampled the pixels surrounding the rounded pixel with
weights proportional to the distance from the point.

I then inverted the matrix of intensities to enable the minimization algorithm
to detect the line. I also added a low-pass filter that could be configured to
drop all pixels below a certain intensity.  

\subsection{Dynamic Programming Search}
I began by creating a simple dynamic programming algorithm based on the one in
the slides provided\cite{slides}. This algorithm attempted to minimise
intensity - finding the white pixels of the original image, and ensure
continuity - producing a line. 

I wrote an energy function based exactly on the slides, it takes in the matrix
of intensities (to allow it to use the intensity for each point) and two points,
and returns a scalar energy value. I created a position function in a similar
manner, this function takes the intensities, the energy matrix and a single
point, to produce a position and energy value, it makes use of the energy
function. The position function is then called from the main dynamic programming
function, for each point.

The algorithm builds up two dimensional energy and position matrices by finding
the minimum energy required to get to each point. The values were  based on the
minimum energy to get to the previous points, and the energy at the point. When
the matrices are full, a contour is produced by back-tracking over the
position matrix. I designed my implementation to be as modular as possible to
enable easy debugging and extension.

Once I had a this solution working, I began extending it to implement the
algorithm defined in the specification\cite{spec}. This required the
energy and position matrices being expanded into the third dimension to
represent all of the possible points that could be the next point in the line.
The energy and position functions each gained an extra parameter, the next point
being considered.  It also meant adding an extra level of looping to the
algorithm as all possibilities of the next point had to be considered as well as
the previous points. This more complex algorithm ensures that straight lines are
favoured.

\section{Results}

\begin{figure}
\centering
\includegraphics[width=\columnwidth]{NotInverted.png}
\caption{The contour from the complex algorithm overlaid on the original image}
\end{figure}


Both of the algorithms were able to produce a contour that matched the outline
of the tongue. The output produced by the two algorithms can be seen in
Appendix \ref{algorithms}. As can be seen, the line produced by the more complex
algorithm gave a smoother line than the simple algorithm.

\subsection{Scaling}
I ran the algorithm for a range of number of divisions (M), I found that the
algorithm doesn't scale well. As the numbers get larger, the amount of time
taken quickly becomes infeasible. Empirically executions took several minutes to
complete, theoretically they would soon take hours to halt. This is expected
since the algorithm contains three nested loops whose execution time is
proportional to the size of M. The expected time complexity is $O(n^3)$, the
graph seen in Appendix \ref{complexityGraph} appears to agree with the theory. 

\subsection{Varying Lambda}
Varying the value of lambda affects how much influence the parts of the
algorithm have on the energy associated with a point. If lambda is small, then
the energy at a given point is most important; if it is large then the
smoothness of the line is more important.

Images in Appendix \ref{lambda} show various levels of lambda. When lambda is zero, the
energy function is simply the intensity of the associated pixel. When lambda is
one, the energy is based entirely on how straight the contour formed by the
three points is. It would seem to be fairly instance specific as to the ideal
value of lambda.

\subsection{Initialisation}
To test how robust the solution is to different initialisations, I began by
switching the order of the two contours. I found that this produced exactly the
same results as before. 

Next I created a contour (shown in Appendix \ref{simpleContour}) by removing points from
the original contours until I was left with just 13 points. I found that this
contour did follow the tongue for part of the image, but didn't pick up on the
curve to the left of the contour. 

I also attempted to initialise the algorithm using contours perpendicular to the
original contours. As expected the simple algorithm was not able to cope with
this and drew a line that didn't fit the contour at all. However, I was 
surprised to find that the complex algorithm actually managed to follow the 
contour for a reasonable distance. These results can be seen in Appendix
\ref{sidewaysContour}

\subsection{Different Images}
I also ran the algorithm on different images. The first of these was an image of
an aircrafts vapour trail\cite{vapourImg} (Appendix \ref{vapour}), in this image
the contour was oriented vertically, running from top to bottom. I found that
the complex algorithm found the line easily. The simple algorithm was able to
pick out the contour after I applied the low pass filter. I believe this is
because the image is two shades of grey rather than black and white; this leads
to energy values that don't have a great enough difference to influence the
contour. 

I then tried a very jagged image of some stairs\cite{stairsImg}. I expected the
algorithms to perform very poorly on this image because they are written to draw
straight lines. For the simple algorithm, I was correct. As can be seen in
Appendix \ref{simpleStairs} a contour was produces that was simply a straight
line running through the middle of the stairs. For the complex algorithm, my
assumptions were incorrect, it created a smooth curving line, containing no
sharp turns, that followed the stairs well. This contour can be seen in Appendix
\ref{complexStairs}.  

\subsection{Sampling Method}
When generating the intensity matrix, the floating point values of the
interpolated points do not match exactly with the integer pixels of the image.
As mentioned above I implemented two different schemes for determining the
intensity. The first method, rounding, worked well for both algorithms. I
predicted that my \em MultiSampling \em method would produce better results as
it took into consideration the intensity of a region so wouldn't be thrown off
by anomalous pixels. I found that although it did produce slightly different
contours, I wasn't able to decide if they were quantitatively better. 

\section{Conclusion}
During this project I have gained a huge amount of experience. I was able to
work with MATLAB for the first time and I now understand how dynamic programming
can be used to detect contours. I feel that I was successful in implementing an
algorithm that detects a contour in an image. 

If I were to extend the project, I would investigate either detecting closed 
contours or detecting multiple contours in a single image. It might also be
interesting to benchmark how slow the brute force method would be to get a
better understanding of how efficient this method is. 



\begin{thebibliography}{9}

	\bibitem{vapourImg}
		Flickr Image of a Vapour Trail, \\
		Charlie Brewer,\\
		\url{http://www.flickr.com/photos/charliebrewer/484937808/}

	\bibitem{stairsImg}
		Flickr Image of Stairs, \\
		judepics, \\
		\url{http://www.flickr.com/photos/judepics/2371279935/}

	\bibitem{slides}
		Dynamic Programming Slides, \\
		Steve Gunn, \\
	\url{https://secure.ecs.soton.ac.uk/notes/comp3032/Lectures/DP.pdf}

	\bibitem{spec}
		Coursework Specification, \\
		Steve Gunn, \\
	\url{https://secure.ecs.soton.ac.uk/notes/comp3032/Assignment/ICE.pdf}

\end{thebibliography}

\onecolumn
\appendix
\appendixpage

\section{Search Space Extraction}
\label{SearchSpaceExt}
\begin{center}
\includegraphics[width=\textwidth]{SearchSpace.png}
\end{center}

This image shows how the search space was created by sampling from interpolated
points on the image.

\section{Dynamic Programming Algorithm Output}

\label{algorithms}

\subsection{Simple Algorithm}
\begin{center}
\includegraphics[width=\textwidth]{SimpleContour.png}
\end{center}

This image was created using the simple dynamic programming algorithm. The image
has been inverted to show how the algorithm minimises the energy.

\subsection{Complex Algorithm}
\begin{center}
\includegraphics[width=\textwidth]{ComplexContour.png}
\end{center}

This image was generated using the complex dynamic programming algorithm.

\section{Time Complexity Graphs}

\subsection{Simple Algorithm}

\begin{center}
\includegraphics[width=\textwidth]{SimpleComplexity.png}
\end{center}

A graph of Number of Divisions (x) vs. Time Taken in Seconds (y) for the simple
algorithm. 
MATLAB's basic fit function suggests that this graph is $n^2$.

\subsection{Complex Algorithm}

\label{complexityGraph}

\begin{center}
\includegraphics[width=\textwidth]{ComplexComplexity.png}
\end{center}

A graph of Number of Divisions (x) vs. Time Taken in Seconds (y) for the complex
algorithm. 
MATLAB's basic fit function suggests that this graph is $n^3$.

\section{Values of Lambda}

\label{lambda}

\subsection{$\lambda = 0$}
\begin{center}
\includegraphics[width=\textwidth]{Lambda0.png}
\end{center}

When lambda is zero, the energy function simply returns the lowest intensity.

\subsection{$\lambda = 0.25$}
\begin{center}
\includegraphics[width=\textwidth]{Lambda025.png}
\end{center}

When lambda is $0.25$, the energy function is influenced more by the intensity
that the smoothness.

\subsection{$\lambda = 0.5$}
\begin{center}
\includegraphics[width=\textwidth]{Lambda05.png}
\end{center}

When lambda is $0.5$ the energy function is influenced equally by the intensity
and the smoothness. 

\subsection{$\lambda = 0.75$}
\begin{center}
\includegraphics[width=\textwidth]{Lambda075.png}
\end{center}

When lambda is $0.75$ the energy function is influences more by the smoothness
than the intensity. 

\subsection{$\lambda = 1$}
\begin{center}
\includegraphics[width=\textwidth]{Lambda1.png}
\end{center}

When lambda is one, the energy function simply returns a straight line. 

\section{Different Initialisation Contours}


\subsection{Simplfied Contour}
\label{simpleContour}

\begin{center}
\includegraphics[width=\textwidth]{SimpleInitialContour.png}
\end{center}

An image showing the complex algorithm run with a contour containing fewer
points that before. 

\subsection{Sideways Contour}
\label{sidewaysContour}

\subsubsection{Simple Algorithm}
\label{simpleSideways}
\begin{center}
\includegraphics[width=\textwidth]{SimpleSideways.png}
\end{center}

An image showing the output of the simple algorithm on a sideways initial
contour.

\subsubsection{Complex Algorithm}
\label{complexSideways}
\begin{center}
\includegraphics[width=\textwidth]{ComplexSideways.png}
\end{center}

An image showing the output of the complex algorithm on a sideways initial
contour.

\section{Different Images}

\subsection{Vapour Trail}
\label{vapour}
\subsubsection{Simple Algorithm, No Low Pass Filter}
\begin{center}
\includegraphics[width=\textwidth]{SimpleVapourNoLow.png}
\end{center}

An image showing the contour produced by the simple algorithm on an image of an
aircrafts vapour trail, without a low pass filter.

\subsubsection{Simple Algorithm, Low Pass Filter}
\begin{center}
\includegraphics[width=\textwidth]{SimpleVapourLow.png}
\end{center}

An image showing the contour produced by the simple algorithm on an image of an
aircrafts vapour trail, with a low pass filter.

\subsubsection{Complex Algorithm}
\begin{center}
\includegraphics[width=\textwidth]{ComplexVapour.png}
\end{center}

An image showing the contour produced by the complex algorithm on an image of an
aircrafts vapour trail.

\subsection{Stairs}

\subsubsection{Simple Algorithm}
\label{simpleStairs}
\begin{center}
\includegraphics[width=\textwidth]{SimpleStairs.png}
\end{center}

An image showing the contour produced by the simple algorithm on an image of
some stairs. 

\subsubsection{Complex Algorithm}
\label{complexStairs}
\begin{center}
\includegraphics[width=\textwidth]{ComplexStairs.png}
\end{center}

An image showing the contour produced by the complex algorithm on an image of
some stairs. 

\end{document}
